\apendice{Documentación técnica de programación}

\section{Introducción}

\section{Estructura de directorios}
Se va a comentar la estructura de ficheros del repositorio de GitHub que contiene todos los materiales relativos al proyecto.

\begin{itemize}
\item
/: Directorio raíz del repositorio.
\item
/docs: Directorio que contiene todos los ficheros relativos a la documentación.
\item
/docs/Latex: Carpeta con todos los ficheros necesarios para la generación de la memoria y los anexos con Latex.
\item
/docs/Latex/img: Contiene todas las imágenes que aparecerán el los ficheros generado en Latex.
\item
/docs/Latex/tex: Contiene los ficheros de Latex secundarios que utilizarán los principales para generar la memoria y los anexos.
\item
/game: Contiene la solución de Visual Studio con el código relativo al vidojuego desarrollado.
\item
/game/Assets: Contiene todos los ficheros de código en C\# y en formatos especializados de Unity necesarios para que Unity pueda exportar el juego correctamente.
\item
/game/Assets/Audio: Ficheros de audio del proyecto.
\item
/game/Assets/Character: Elementos estéticos exclusivos del Player.
\item
/game/Assets/Character/Animations: Animaciones que provee Platformer Microgame utilizadas en el Player.
\item
/game/Assets/Character/Sprites: Sprites del Player.
\item
/game/Assets/Environment: Sprites utilizados para crear los mapas de los niveles jugables.
\item
/game/Assets/ModAssets: Elementos estéticos extra que provee Platformer Microgame para ofrecer más variedad estética.
\item
/game/Assets/Prefabs: Contiene todos los prefabs del proyecto.
\item
/game/Assets/Scenes: Contiene todas las escenas que conforman el proyecto.
\item
/game/Assets/Scripts: Ficheros de C\# utilizados en el proyecto.
\item
/game/Assets/Scripts/Animation: Ficheros de C\# utilizados en el proyecto relativos a la animación de sprites.
\item
/game/Assets/Scripts/Core: Ficheros de C\# utilizados en el proyecto. Contiene las clases encargadas del funcionamiento básico de los niveles.
\item
/game/Assets/Scripts/Gameplay: Ficheros de C\# utilizados en el proyecto. Contiene las clases que son eventos de Simulation.
\item
/game/Assets/Scripts/GizmosUI: Ficheros de C\# utilizados en el proyecto. Contiene las clases utilizadas para mostrar los Gizmos en el editor.
\item
/game/Assets/Scripts/Mechanics: Ficheros de C\# utilizados en el proyecto. Contiene las clases que implementa las mecánicas del juego.
\item
/game/Assets/Scripts/Model: Ficheros de C\# utilizados en el proyecto. Contiene una clase con las variables que se consultarán durante los niveles.
\item
/game/Assets/Scripts/Sound: Ficheros de C\# utilizados en el proyecto. Contiene las clases encargadas de gestionar el audio.
\item
/game/Assets/Scripts/UI: Ficheros de C\# utilizados en el proyecto. Contiene todas las clases relativas a la interfaz de usuario.
\item
/game/Assets/Scripts/View: Ficheros de C\# utilizados en el proyecto. Contiene clases que ofrecen comportamientos que afectan de forma especial a como debería verse un objeto.
\item
/game/Assets/Sprites: Sprites del proyecto que no pertenecen a Platformer Microgame.
\item
/game/Assets/Scripts: Elementos usados para generar mapas mediante la herramienta Tilemap de Unity.
\end{itemize}

\section{Manual del programador}

\section{Compilación, instalación y ejecución del proyecto}

\section{Pruebas del sistema}
