\capitulo{3}{Conceptos teóricos}

\section{Unity \cite{Unity}}
En este apartado se va a explicar que es Unity y algunos de los elementos que lo componen y de los que hace uso. Este apartado es importante porque durante toda la memoria se va a hacer referencia a estos elementos asumiendo que se ha leído este apartado.

Unity(1) es un motor gráfico orientado al desarrollo de videojuegos que permite desarrollar para ordenador (Microsoft Windows, Mac OS y Linux), consolas, móvil y dispositivos de realidad aumentada.

Unity tiene una serie de elementos, pero los más destacables son:

\subsection{MonoBehaviour \cite{ClaseMonobehaviour}}
MonoBehaviour es la clase de la que (en principio) parten todas las clases que utiliza Unity. Esta clase y sus hijas son las que inicializa Unity al crear una escena. La razón por la que esta clase es tan importante y requiere de explicación son los siguientes métodos:

\textbf{Awake() y Start():} Tanto Awake como Start son métodos que se ejecutan al crear un objeto que herede de MonoBehaviour, sin embargo funcionan de manera ligeramente distinta. 

El método Awake se llama en el instante exacto en el que se carga el script al que está asociado o en el que se crea la instancia del nuevo objeto de esta clase. Funciona prácticamente como el constructor de una clase. Es un método muy adecuado para inicializar variables ya que se ejecutará justo al crearse la instancia del objeto o cargarse el script. 

El método Awake se ejecutará siempre independientemente de que el objeto este activado o no. Con activado se hace referencia al atributo booleano enabled de la clase MonoBehaviour. Cuando este atributo este a false el objeto no realizará ninguna operación (actuará como si estuviese desactivado) y cuando este a true funcionará con normalidad.

Como se ha mencionado el método Start funciona de manera muy similar a Awake pero con dos diferencias clave. El método Start se activa antes de que se llame a cualquier método Update (explicados a continuación), pero no garantiza que se valla a llamar en el mismo momento en el que se crea el objeto o se carga el script (a diferencia del método Awake). La otra diferencia con Awake es que el método Start se llamará solo si el objeto esta activado (enabled a true) mientras que Awake se llamará siempre.\\


\textbf{Update(), FixedUpdate() y LateUpdate():} Estos tres son métodos que se llaman repetitivamente hasta la desaparición del objeto. Estos métodos son los que utilizará Unity para saber en todo momento que es lo que tiene que hacer y en qué momento ha de hacerlo. Pero ¿Por qué hacer tres métodos distintos para albergar instrucciones que se repetirán continuamente? La razón de esto reside en cuándo se ejecutan estos tres métodos.

El método Update se llama cada frame. Explicado a grandes rasgos un frame representa el momento en el que cambia lo que sale por pantalla. Así que cada vez que cambia lo que se ve en pantalla se llama al método Update. El tiempo que puede pasar entre frames no tiene por qué ser siempre el mismo. Es por ello que el periodo entre llamadas al método Update no siempre será el mismo.

El método FixedUpdate se llama repetitivamente, con la excepción de que no lo hace cada frame, sino que se repite en el tiempo de manera regular. El periodo entre llamadas al método FixedUpdate será siempre de 0.02 segundos. Este periodo se puede modificar pero la llamada al método Fixed Update siempre será regular. Esto hace al método FixedUpdate un método ideal para realizar cálculos dependientes del momento del tiempo en el que te halles (calcular trayectorias de objetos por ejemplo).

El método LateUpdate es muy similar al método Update. También se ejecuta en cada frame pero difiere del método Update en que se ejecutará siempre después de los demás métodos “update”. Esto puede ser útil para la actualización de elementos que requieren que se hayan hecho una serie de cambios antes. Un ejemplo podría ser el seguimiento de una cámara a un objeto. Si se establece el seguimiento de la cámara y luego se actualiza la posición del objeto la cámara va a estar persiguiendo al objeto en una posición en la que no está.\\


\textbf{Mensajes:} Además de estos métodos hay otros métodos que se identifican por el nombre de “mensajes”. Los “mensajes” funcionan de forma un poco diferente a los métodos normales y corrientes. Son métodos que se pueden activar con normalidad al llamarlos, pero también son métodos que se pueden activar al “recibir un mensaje” utilizando el método SendMessage() de la clase GameObject (esta clase se explicará a continuación). La gracia de estos métodos “mensaje” no está en que se encuentren en la clase MonoBehaviour, sino como interactúan con otras clases y objetos. Es por eso que a continuación se mencionarán los más interesantes, pero la explicación de los métodos se realizará en otro apartado en el que se haga referencia a estos métodos y se comprenda mejor el uso de estos.

Métodos “mensaje”:
\begin{itemize}
\item
OnCollisionEnter.
\item
OnCollisionEnter2D.
\item
OnCollisionExit.
\item
OnCollisionExit2D.
\item
OnCollisionStay.
\item
OnCollisionStay2D.
\item
OnTriggerEnter.
\item
OnTriggerEnter2D.
\item
OnTriggerExit.
\item
OnTriggerExit2D.
\item
OnTriggerStay.
\item
OnTriggerStay2D.
\item
Start (también es un método “mensaje”).
\item
Update (también es un método “mensaje”).
\end{itemize}

\subsection{GameObject \cite{ClaseGameObject}}
La clase GameObject es la clase de la que parten todos los objetos que va a utilizar Unity. Todos los elementos que creas en Unity son GameObject. La clase GameObject por defecto añade atributos que ofrecen información básica sobre ese GameObject. Algunos de los más importantes serían: transform para saber qué región del espacio ocupa el GameObject, tag para identificar al GameObject y name para identificar tanto al GameObject como a los componentes de este (todos comparten el mismo nombre).
GameObject por si sola es una clase inútil. Pero lo importante de GameObject es la capacidad que posee para añadirse componentes a si mimo (con el método GetComponent entre otros). Los componentes, al ser añadidos a la instancia de GameObject que le corresponda ya pueden ser usados por Unity. Los componentes de GameObject pueden ser objetos de cualquier tipo, sin necesidad de heredar de MonoBehaviour.

GameObject posee varios métodos estáticos, pero se van a explicar dos que se creen dignos de mención. Estos métodos son Destroy e Instantiate. Lo que hacen estos métodos se puede deducir por el nombre de estos. Destroy destruye instancias de GameObject y/o componentes de estos y el método Instantiate las crea. Sn embargo, cabe mencionar que el método Instantiate no crea de verdad los objetos, sino que clona uno existente y devuelve el objeto clonado. Esto es importante porque Unity provee unos objetos especiales que son los Prefabs. Estos Prefabs son un GameObject persistente que guarda la configuración con la que ha sido construido ese GameObject. De esta manera si tienes un Prefab de, por ejemplo, un enemigo del juego puedes crear todas las copias de ese enemigo que quieras llamando al método Instantiate y pasando como argumento ese Prefab. Con los Prefabs se logra tener copias idénticas de objetos sin tener que crearlas manualmente cada vez.

La última característica importante de GameObject es el método SendMessage. Este método permite al GameObject o un componente suyo mandar un mensaje a los demás componentes del GameObject, haciendo que todos los componentes (en realidad solo los que hereden de MonoBehaviour) que tengan un método con nombre igual al pasado como argumento en el método SendMessage ejecutarán ese método. Un ejemplo sería gameobject.SendMessage(“Metodo1”). Al ejecutar ese método se hará que todos los componentes que hereden de MonoBehaviour y tengan un método llamado Metodo1 invoquen ese método.

\subsection{Escenas}
Las escenas son elementos de Unity que contienen otros objetos. Una escena puede ser un nivel o un menú del juego.

\subsection{Físicas en Unity}
Un objeto por defecto no se ve afectado por las físicas. Sin embargo añadiendo el componente Rigidbody (Rigidbody2D para los juegos en dos dimensiones) el GameObject al que este asociado variará su posición como si estuviese afectado por las físicas.

Rigidbody tiene un atributo llamado velocity. Este atributo un vector de 3 dimensiones que representa en qué dirección se moverá el GameObject afectado por las físicas. Un componente Rigidbody hace que su GameObject se vea afectado por la gravedad o no con el atributo booleano useGravity.

Rigidbody tiene un atributo booleano llamado isKinematic. Este atributo hace que un objeto no se vea afectado por las colisiones. Un objeto con un Rigidbody con el atributo isKinematic igual a true (objeto 1) que colisiona con otro con un Rigidbody con el atributo iskinematic igual a false (objeto 2) provoca que el objeto 1 modifique su movimiento teniendo en cuenta las reacciones físicas generadas por la colisión con el objeto 2. El objeto 2, sin embargo no verá su movimiento afectado por la colisión.

\subsection{Colisión entre objetos}
Unity ofrece por defecto una manera de manejar las colisiones entre objetos. Para que un GameObject ofrezca la posibilidad de colisionar con otro objeto, este debe de tener un componente llamado Collider (para los juegos en dos dimensiones Collider2D). Este componente te permite determinar una región del espacio en la que otro GameObject con un componente Collider se considerará colisionando con el primer objeto. Si dos Colliders comparten algún punto de los espacios que delimitan, se considerará que se ha producido una colisión entre los GameObject a los que pertenecen. 

La región del espacio que ocupa un Collider por defecto es estática y no se puede mover, sin embargo moviendo la posición del GameObject moverá la posición del Collider, pues su posición es relativa al GameObject.

La colisión entre objetos se desarrolla mediante la ejecución de los métodos OnCollisionEnter/Stay/Exit de los componentes que heredan de la clase MonoBehaviour que tengan implementados esos métodos. 
Si dos objetos tienen un Rigidbody y un Collider, al colisionar ambos objetos variaran su movimiento en consecuencia. Sin embargo, si se desea evitar esto, el componente Collider tiene un atributo booleano llamado isTrigger. Si está a false el funcionamiento será el explicado anteriormente. Sin embargo, si está a true el atributo isTrigger, su movimiento no se verá afectado al colisionar pero sí que se detectará la colisión. En caso de que el atributo isTrigger sea true, la colisión no se resolverá ejecutando los métodos anteriormente explicados, sino que se resolverá mediante los métodos OnTriggerEnter/Stay/Exit de los componentes que heredan de la clase MonoBehaviour que tengan implementados esos métodos.
