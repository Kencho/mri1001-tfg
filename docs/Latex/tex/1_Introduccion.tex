\capitulo{1}{Introducción}

Los videojuegos son productos software enfocados al ocio. Desde el primer videojuego, el \textit{Tennis for Two \cite{Tennis4two}} creado en 1958 por el físico William Higginbotham hasta los juego actuales, estos han ido creciendo en calidad y envergadura hasta convertirse la industria de los videojuegos en una de las que más dinero mueve habiendo recaudado en 2020 entorno a 140.000 millones de dolares \cite{2020GamesMoney}.

Sin embargo, los videojuegos son productos software muy particulares intrínsecamente relacionados con otros ámbitos del ocio como la música o el arte visual. Debido a la curiosidad que genera este producto tan complejo y el amor que se profesa por los videojuegos se ha decidido desarrollar un videojuego con el objetivo de entender cual es el funcionamiento interno de un videojuego y como es su desarrollo como producto software.

Se plantearán las mecánicas y elementos que tendrá el videojuego. Una vez decidido esto, se generará un documento de diseño de juego con toda la información relativa al producto que se desea lograr y de lo que estará compuesto. El documento de diseño de juego tendrá un lenguaje coloquial pero su importancia radica en lo útil que resulta para marcar el rumbo del proceso de creación del juego y ofrecer una visión de lo que acabará siendo el videojuego.

Con el documento de diseño de juego creado se procederá a desarrollar el videojuego con el objetivo de alcanzar las metas establecidas en este.

\section{Materiales Adjuntos}
Este TFG consta de los siguientes materiales:
\begin{itemize}
\item
La memoria del TFG.
\item
Los anexos (el documento de diseño de juego estará redactado en los anexos).
\item
La solución de Visual Studio que contiene el código en C\# correspondiente al videojuego.
\item
El fichero comprimido con el sistema de ficheros y el ejecutable que hará funcionar el juego en ordenador.
\item
El fichero comprimido con el sistema de ficheros y el ejecutable que permita ejecutar el juego en WebGL.
\end{itemize}

Todos estos elementos serán accesibles a través del repositorio de GitHub: \url{https://github.com/Kencho/mri1001-tfg}

\section{Estructura de la memoria}
La memoria consta de los siguientes apartados:
\begin{itemize}
\item
\textbf{Introducción:} Resumen del motivo de este proyecto, materiales que conformará el producto y documentos relativos a este.
\item
\textbf{Objetivos del proyecto:} Explicación de los objetivos propuestos para el proyecto.
\item
\textbf{Conceptos teóricos:} Conocimiento teórico relativo al proyecto que introduce conceptos mencionados en los demás apartados que se darán por aprehendidos.
\item
\textbf{Técnicas y herramientas:} Técnicas y herramientas relativas al desarrollo del proyecto, en qué consisten y qué uso se les va a dar.
\item
\textbf{Aspectos relevantes del desarrollo:} Explicación de la implementación que se ha realizado sobre los elementos del videojuego.
\item
\textbf{Trabajos relacionados:} Proyectos relativos al videojuego que se está desarrollando, tanto como referencias como proyectos en los que inspirarse.
\item
\textbf{Conclusiones y líneas de trabajo futuras:} Conocimientos adquiridos, críticas a las herramientas utilizadas y próximas tareas relativas al proyecto.
\end{itemize}

\section{Anexos}
La memoria contará adicionalmente con los siguientes anexos:
\begin{itemize}
\item
\textbf{Plan de proyecto software:} Planificación temporal del desarrollo del proyecto y estudio de viabilidad de este.
\item
\textbf{Especificación de requisitos:} Especificación de requisitos y casos de uso.
\item
\textbf{Especificación de diseño:} Elementos relativos al diseño.
\item
\textbf{Documentación técnica de programación:} Estructura de ficheros y elementos de interés relativo al programador.
\item
\textbf{Documentación de usuario:} Información relativa a los usuarios de la aplicación y manual del usuario.
\item
\textbf{Documento de diseño del juego:} Documento de diseño del juego con la información relativa al videojuego que se desea desarrollar.
\end{itemize}