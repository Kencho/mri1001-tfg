\capitulo{1}{Introducción}

Los videojuegos son productos software enfocados al ocio. Desde el primer videojuego, el \textit{Tennis for Two \cite{Tennis4two}} creado en 1958 por el físico William Higginbotham hasta los juego actuales, estos han ido creciendo en calidad y envergadura hasta convertirse la industria de los videojuegos en una de las que más dinero mueve habiendo recaudado en 2020 entorno a 140.000 millones de dolares \cite{2020GamesMoney}.

Sin embargo, los videojuegos son productos software muy particulares intrínsecamente relacionados con otros ámbitos del ocio como la música o el arte visual. Debido a la curiosidad que genera este producto tan complejo y el amor que se profesa por los videojuegos se ha decidido desarrollar un videojuego con el objetivo de entender cual es el funcionamiento interno de un videojuego y como es su desarrollo como producto software.

Se plantearán las mecánicas y elementos que tendrá el videojuego. Una vez decidido esto, se generará un documento de diseño de juego con toda la información relativa al producto que se desea lograr y de que estará compuesto. El documento de diseño de juego tendrá un lenguaje coloquial pero su importancia radica en lo útil que resulta para marcar el rumbo del proceso de creación del juego y ofrecer una visión de lo que acabará siendo el videojuego.

Con el documento de diseño de juego creado se podrá proceder a desarrollar el videojuego con el objetivo de alcanzar las metas establecidas en este.

\section{Materiales Adjuntos}
Este TFG consta de los siguientes materiales:
\begin{itemize}
\item
La memoria del TFG.
\item
Los anexos (el documento de diseño de juego estará redactado en los anexos).
\item
La solución de Visual Studio que contiene el código en C\# correspondiente al videojuego.
\item
El fichero comprimido con el sistema de ficheros y el ejecutable que hará funcionar el juego en ordenador.
\item
El fichero comprimido con el sistema de ficheros y el ejecutable que permita ejecutar el juego en WebGL.
\end{itemize}

Todos estos elementos serán accesibles a través del repositorio de GitHub: \url{https://github.com/Kencho/mri1001-tfg}
